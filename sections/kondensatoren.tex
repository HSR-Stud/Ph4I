\section{Kondensatoren}

Kapazität:
\[
	C = \frac{Q}{U} \quad \left[F\right]
\]

Kapazität Plattenkondensator:
\[
	C = \frac{\varepsilon_0 A}{d} \quad \left[F\right]
\]
$d$: Abstand zwischen Platten\\
$A$: Fläche einer Platte

Energie des geladenen Kondensators:
\[
	E = \frac{CU^2}{2} = \frac{Q^2}{2C} = \frac{QU}{2} \quad \left[J\right]
\]

\subsection{Parallel-Schaltung}

Kapazitäten, Spannungen und Energie addieren sich. Spannung und Energie verteilt
sich proportional zur Kapazität der Kondensatoren.

\begin{minipage}{.5\linewidth}
	\begin{circuitikz}

\draw (1.5,0.5) -- (0,0.5)
	to[C, l=$C_1$] (0,2.5) -- (3,2.5)
	to[C, l=$C_2$] (3,0.5) -- (1.5,0.5) -- (1.5,0) -- (5,0)
	to[battery] (5,3) -- (1.5,3) -- (1.5,0.5)
;

\draw (0,2.5) node[left]{$A$};
\draw (0,0.5) node[left]{$B$};
\draw (3,2.5) node[right]{$C$};
\draw (3,0.5) node[right]{$D$};

\draw (5,2) node[right]{$+$};
\draw (5,1) node[right]{$-$};

\draw (0.5,1.5) node[right]{$U_1$};
\draw (2.5,1.5) node[left]{$U_2$};
\draw (5.5,1.5) node[right]{$C$};

\end{circuitikz}

\end{minipage}
\begin{minipage}{.5\linewidth}
	\begin{align*}
		& C = C_1 + C_2 \\
		& U = U_1 + U_2 \\
		& E = E_1 + E_2 \\
		& \frac{U_1}{C_1} = \frac{U_2}{C_2} \\
		& \Phi_A = \Phi_C \\
		& \Phi_B = \Phi_D \\
	\end{align*}
\end{minipage}

\subsection{Durchschlagsfestigkeit}

Die Durchschlagsfestigkeit ist die maximal mögliche Spannung pro Distanz
zwischen zwei unterschiedlich geladenen Objekten (zB Kondensatorplatten). Wird
die Spannung überschritten oder die Distanz unterschritten, geschieht ein
Ladungsaustausch ($\rightarrow$ es "`blitzt und tätscht"').

\subsection{Dielektrika}

Kapazität des Plattenkondensators mit Dielektrikum:
\[
	C = \varepsilon_r \varepsilon_0 \cdot \frac{A}{d} = \varepsilon \cdot \frac{A}{d}
\]
