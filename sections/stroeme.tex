\section{Elektrische Ströme}

\textbf{Elektrischer Strom} ist die pro Zeiteinheit durch einen elektrischen Leiter
fliessende Ladung.

TODO formel

\textbf{Stromdichte}:
\begin{align*}
	& j \approx \frac{\textrm{Strom}}{\textrm{Fläche}} \approx \vec{E} \\
	& j = \rho \cdot E = \frac{1}{\rho}E
\end{align*}

\textbf{Elektrische Leitfähigkeit}:
\begin{align*}
	& j = \sigma \cdot \vec{E} \\
	& \sigma = \frac{j}{\vec{E}}
\end{align*}

\textbf{Spezifischer Widerstand}:
\begin{align*}
	& \rho = \frac{1}{\sigma}
\end{align*}

\subsection{Widerstände}

\subsubsection{Innenwiderstand}

Innenwiderstand einer Stromquelle:
\begin{align*}
	& R_i = R \left(\frac{U_0}{U} - 1\right)
\end{align*}

$U_0$: Ohne Last\\
$U$: Mit Last

\subsubsection{Schaltungen}

\begin{minipage}[t]{.5\linewidth}
	Seriell:
	\begin{align*}
		& I = I_1 = I_2 = I_3 \\
		& R = R_1 + R_2 + R_3 \\
		& U = U_1 + U_2 + U_3
	\end{align*}
	\begin{circuitikz}

\draw (0,0)
	to[battery] (0,2) -- (6,2) -- (6,0)
	to[R, l=$R_3$] (4,0)
	to[R, l=$R_2$] (2,0)
	to[R, l=$R_1$] (0,0)
;

\end{circuitikz}

\end{minipage}
\begin{minipage}[t]{.5\linewidth}
	Parallel:
	\begin{align*}
		& I = I_1 = I_2 = I_3 \\
		& 1/R = 1/R_1 + 1/R_2 + 1/R_3 \\
		& U = U_1 = U_2 = U_3
	\end{align*}
	\begin{circuitikz}

\draw (0,0) -- (6,0)
	to[R, l=$R_3$] (6,2) -- (0,2)
	to[battery] (0,0)
;

\draw (2,0) to[R, l=$R_1$] (2,2);
\draw (4,0) to[R, l=$R_2$] (4,2);

\end{circuitikz}

\end{minipage}


\subsection{Kirchhoffsche Regeln}

\subsubsection{Knotenregel}

In einem Knotenpunkt eines elektrischen Netzwerkes ist die Summe der
zufliessenden Ströme gleich der Summe der abfliessenden Ströme.
\[
	\sum_{k=1}^n I_k = 0
\]

\subsubsection{Maschenregel}

Alle Teilspannungen eines Umlaufs bzw. einer Masche in einem elektrischen
Netzwerk addieren sich zu null. Die Richtung des Umlaufes kann beliebig gewählt
werden; sie legt dann aber die Vorzeichen der Teilspannungen fest.
\[
	\sum_{k=1}^n U_k = 0
\]
