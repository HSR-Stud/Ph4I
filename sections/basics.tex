\section{Basics}

\subsection{Elektrische Ladung}

Die Ladung 1 Coulomb (C) entspricht der Ladung von $\approx 6.24 \cdot 10^{18}$
Elektronen.

In einem abgeschlossenen System bleibt die Summe aller Ladungen konstant.

\subsection{Coulombsches Gesetz}

Die Kraft $F$ zwischen zwei punktförmigen Ladungen $Q_1$ und $Q_2$ ist umgekehrt
proportional zum Quadrat ihres Abstandes $r$.
\[
	F = a \frac{Q_1Q_2}{r^2}
\]
Dabei ist $a$ eine Konstante die von der Wahl der Ladungseinheit abhängt. Im SI
System ist $a$ folgendermassen festgelegt:
\[
	a = \frac{1}{4 \pi \varepsilon_o}
\]
Hier wiederum ist $\varepsilon_0$ die \textbf{elektrische Feldkonstante}.
\[
	\varepsilon_0 = 8.854 \cdot 10^{-12}
	\left[ \frac{\textrm{C}}{\textrm{V}\cdot \textrm{m}} \right]
\]
Damit erhält das Coulombsche Gesetz folgende Form:
\[
	F = \frac{1}{4\pi\varepsilon_0} \cdot \frac{|Q_1Q_2|}{r^2}
\]
