\section{Schwingungen}

\subsection{Trägheitsmomente}

Brett: $\displaystyle\frac{m}{12} \left(h^2 + b^2\right)$

Stab: $\displaystyle\frac{m}{12} l^2$

Für weitere Trägheitsmomente, siehe \textit{Taschenbuch der Physik} von Horst
Kuchling, Seite 131-132.

\subsection{Pendelschwingungen}

Pendelschwingungen:
\[
	T = 2 \pi \sqrt{\frac{J_S + ma^2}{mga}}	
	\quad \left[ s \right]
\]
Dabei ist $T$ die Schwingungsdauer, $J_S$ das Trägheitsmoment und $a$ der
Abstand vom Schwerpunkt.

Die Schwingungsdauer $T$ entspricht der reziproken Frequenz $1/f$. Die
Kreisfrequenz $\omega$ berechnet sich also wie folgt:
\[
	\omega = 2\pi\frac{1}{T}
\]
