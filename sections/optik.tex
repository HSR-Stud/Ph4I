\section{Optik}

\subsection{Reelle / Virtuelle Bilder}

Ein \textbf{reelles Bild} existiert "wirklich", d.h. von dem Ort des Bildes
gehen wirklich Lichtstrahlen aus oder die von einem Objektpunkt ausgehenden
Strahlen haben sich dort getroffen und gehen nun wieder auseinander. Beispiele:
Gemälde, Projektionen, Leinwände oder Fernsehbilder.

Ein \textbf{virtuelles Bild} kann im Gegensatz zu einem reellen Bild nicht auf
einem Schirm abgebildet werden. Es entsteht, wenn sich das Objekt zwischen
Brennpunkt und Linse befindet. Beispiele: Spiegelbilder, Lupe, Mikroskop.
