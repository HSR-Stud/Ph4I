\section{Optik}

\subsection{Reelle / Virtuelle Bilder}

Ein \textbf{reelles Bild} existiert "wirklich", d.h. von dem Ort des Bildes
gehen wirklich Lichtstrahlen aus oder die von einem Objektpunkt ausgehenden
Strahlen haben sich dort getroffen und gehen nun wieder auseinander. Beispiele:
Gemälde, Projektionen, Leinwände oder Fernsehbilder.

Ein \textbf{virtuelles Bild} kann im Gegensatz zu einem reellen Bild nicht auf
einem Schirm abgebildet werden. Es entsteht, wenn sich das Objekt zwischen
Brennpunkt und Linse befindet. Beispiele: Spiegelbilder, Lupe, Mikroskop.

\subsection{Abbildungsgleichungen}

Gegeben:

\begin{tabular}{lll}
	$b$: Bildweite & $g$: Gegenstandsweite & $f$: Brennweite \\
	$B$: Bildgrösse & $G$: Gegenstandsgrösse & $F$: Brennpunkt
\end{tabular}

Abstandsverhältnisse:
\[
	\frac{1}{f} = \frac{1}{g} + \frac{1}{b}
\]
Grössenverhältnisse:
\[
	\frac{B}{G} = \frac{b}{g}
\]
Abbildungsmassstab $\beta$:
\[
	\beta = \frac{b}{g}
\]

\subsection{Brechungsgesetz}

Gegeben:

\begin{tabular}{ll}
	$\alpha$: Einfallswinkel & $\beta$: Ausfallswinkel \\
	$c_1$: Ausbreitungsgeschwindigkeit Medium 1 & $c_2$: Ausbreitungsgeschwindigkeit Medium 2 \\
	$n_1$: Brechzahl Medium 1 & $n_2$: Brechzahl Medium 2
\end{tabular}

Brechungsgesetz:
\[
	n_1 \sin \alpha = n_2 \sin \beta
\]
\[
	\frac{\sin \alpha}{\sin \beta} = \frac{c_1}{c_2} = \frac{n_2}{n_1}
\]

\subsection{Totalreflektion}

Wenn ein Lichtstrahl aus einem optisch dichteren in ein optisch weniger dichtes
Material übertritt und der Ausfallswinkel gleich oder grösser als 90° ist, so
wird der Lichtstrahl vollständig gemäss Reflektionsgesetz ($\alpha = \beta$)
reflektiert.
